%%%% Proceedings format for most of ACM conferences (with the exceptions listed below) and all ICPS volumes.
\documentclass[sigconf]{acmart}
%%%% As of March 2017, [siggraph] is no longer used. Please use sigconf (above) for SIGGRAPH conferences.

%%%% Proceedings format for SIGPLAN conferences 
% \documentclass[sigplan, anonymous, review]{acmart}

%%%% Proceedings format for SIGCHI conferences
% \documentclass[sigchi, review]{acmart}

%%%% To use the SIGCHI extended abstract template, please visit
% https://www.overleaf.com/read/zzzfqvkmrfzn

\usepackage{booktabs} % For formal tables


% Copyright
%\setcopyright{none}
%\setcopyright{acmcopyright}
%\setcopyright{acmlicensed}
\setcopyright{rightsretained}
%\setcopyright{usgov}
%\setcopyright{usgovmixed}
%\setcopyright{cagov}
%\setcopyright{cagovmixed}


% DOI
\acmDOI{10.475/123_4}

% ISBN
\acmISBN{123-4567-24-567/08/06}

%Conference
\acmConference[WOODSTOCK'97]{ACM Woodstock conference}{July 1997}{El
  Paso, Texas USA} 
\acmYear{1997}
\copyrightyear{2016}

\acmArticle{4}
\acmPrice{15.00}

% These commands are optional
%\acmBooktitle{Transactions of the ACM Woodstock conference}
%\editor{Jennifer B. Sartor}
%\editor{Theo D'Hondt}
%\editor{Wolfgang De Meuter}
\graphicspath{{./images/}}

\begin{document}
\title{SIG Proceedings Paper in LaTeX Format}
%\titlenote{Produces the permission block, and copyright information}
%\subtitle{Extended Abstract}
%\subtitlenote{The full version of the author's guide is available as \texttt{acmart.pdf} document}

\author{Qianqian Zhu}
\affiliation{%
  \institution{Delft University of Technology}
  \city{Delft} 
  \country{Netherlands}
}
\email{qianqian.zhu@tudelft.nl}

\author{Andy Zaidman}
\affiliation{%
  \institution{Delft University of Technology}
  \city{Delft} 
  \country{Netherlands}
}
\email{a.e.zaidman@tudelft.nl}


\begin{abstract}
This paper provides a sample of a \LaTeX\ document which conforms,
somewhat loosely, to the formatting guidelines for
ACM SIG Proceedings.
\end{abstract}

%
% The code below should be generated by the tool at
% http://dl.acm.org/ccs.cfm
% Please copy and paste the code instead of the example below. 
%
\begin{CCSXML}
<ccs2012>
<concept>
<concept_id>10011007.10011074.10011099.10011102.10011103</concept_id>
<concept_desc>Software and its engineering~Software testing and debugging</concept_desc>
<concept_significance>500</concept_significance>
</concept>
</ccs2012>
\end{CCSXML}

\ccsdesc[500]{Software and its engineering~Software testing and debugging}


\keywords{ACM proceedings, \LaTeX, text tagging}

\maketitle

\section{Introduction}

Mutation testing is a fault-based testing technique that has been very actively investigated by researchers since the 1970s. Mutation testing introduces small syntactic changes into the program to generate faulty versions (mutants) according to well-defined rules (mutation operators)~\cite{offutt2011mutation}. Then the quality of a test suite can be qualified as the percentage of mutants it distinguishes from the original program (mutation score). The benefits of mutation testing have been shown in many empirical studies, e.g.~\cite{mathur1994empirical, andrews2005mutation}. %: (1) better fault exposing capability compare to other test coverage criteria, e.g. all-use~\cite{mathur1994empirical,frankl1997all,li2009experimental}; (2) a good alternative to real faults which can provide a good indication of the fault detection ability of a test suite (e.g. Andrews et al.~\cite{andrews2005mutation}). \ANNIBALE{What is the difference between the two points? They seem quite equivalent (or perhaps I missed something).}

Despite its well-known advantages, mutation testing is currently not widely applied.
This is due to the computational cost incurred from executing each mutation against the test suite 
to obtain the mutation score and the number of mutants increasing dramatically with the size of the program. 

To address these limitations, several methods have been proposed in literature, such as mutant sampling~\cite{acree1980mutation} and selective mutation~\cite{offutt1996experimental}. Differently from the aforementioned methods that are independent of the program under test, other procedures have been developed  
to further optimise the mutation execution procedure given the program under test. State-of-the-art techniques falling into this category filter unnecessary executions based on the dynamic information at run-time, e.g., line coverage~\cite{schuler2009javalanche} and state infection~\cite{just2014efficient}. 

In this paper, we further optimise mutation execution using data compression techniques based on state infection. In addition to filtering out unnecessary test executions, we ``compress" mutation execution by selecting a subset of mutants and a subset of test cases to estimate the mutation score with minimal loss of precision. We coined our method ``ComMT", which is short for \underline{Com}pressed \underline{M}utation \underline{T}esting. 

\begin{table}
  \caption{Frequency of Special Characters}
  \label{tab:freq}
  \begin{tabular}{ccl}
    \toprule
    Non-English or Math&Frequency&Comments\\
    \midrule
    \O & 1 in 1,000& For Swedish names\\
    $\pi$ & 1 in 5& Common in math\\
    \$ & 4 in 5 & Used in business\\
    $\Psi^2_1$ & 1 in 40,000& Unexplained usage\\
  \bottomrule
\end{tabular}
\end{table}

\begin{figure}
\includegraphics{fly}
\caption{A sample black and white graphic.}
\end{figure}


Here is a theorem:
\begin{theorem}
  Let $f$ be continuous on $[a,b]$.  If $G$ is
  an antiderivative for $f$ on $[a,b]$, then
  \begin{displaymath}
    \int^b_af(t)\,dt = G(b) - G(a).
  \end{displaymath}
\end{theorem}

Here is a definition:
\begin{definition}
  If $z$ is irrational, then by $e^z$ we mean the
  unique number that has
  logarithm $z$:
  \begin{displaymath}
    \log e^z = z.
  \end{displaymath}
\end{definition}

The pre-defined theorem-like constructs are \textbf{theorem},
\textbf{conjecture}, \textbf{proposition}, \textbf{lemma} and
\textbf{corollary}.  The pre-defined de\-fi\-ni\-ti\-on-like constructs are
\textbf{example} and \textbf{definition}.  You can add your own
constructs using the \textsl{amsthm} interface~\cite{Amsthm15}.  The
styles used in the \verb|\theoremstyle| command are \textbf{acmplain}
and \textbf{acmdefinition}.

Another construct is \textbf{proof}, for example,

\begin{proof}
  Suppose on the contrary there exists a real number $L$ such that
  \begin{displaymath}
    \lim_{x\rightarrow\infty} \frac{f(x)}{g(x)} = L.
  \end{displaymath}
  Then
  \begin{displaymath}
    l=\lim_{x\rightarrow c} f(x)
    = \lim_{x\rightarrow c}
    \left[ g{x} \cdot \frac{f(x)}{g(x)} \right ]
    = \lim_{x\rightarrow c} g(x) \cdot \lim_{x\rightarrow c}
    \frac{f(x)}{g(x)} = 0\cdot L = 0,
  \end{displaymath}
  which contradicts our assumption that $l\neq 0$.
\end{proof}

\appendix
\begin{acks}
  The authors would like to thank Dr. Yuhua Li for providing the
  matlab code of  the \textit{BEPS} method. 

  The authors would also like to thank the anonymous referees for
  their valuable comments and helpful suggestions. The work is
  supported by the \grantsponsor{GS501100001809}{National Natural
    Science Foundation of
    China}{http://dx.doi.org/10.13039/501100001809} under Grant
  No.:~\grantnum{GS501100001809}{61273304}
  and~\grantnum[http://www.nnsf.cn/youngscientsts]{GS501100001809}{Young
    Scientsts' Support Program}.

\end{acks}


\bibliographystyle{ACM-Reference-Format}
\bibliography{mybib} 

\end{document}
